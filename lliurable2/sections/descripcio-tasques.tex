\section{Descripció de tasques}
\subsection{Aspectes legals de la signatua electrònica}
La idea d'aquesta tasca, és la de posar en context el projecte en tot al que fa referència a consentiments informats, validesa de signatures electròniques, què s'ha de tenir en compte per tal de que no es pugui repudiar un document, etc etc.

\subsection{Estudi tecnològic}
Un cop entes el context en el qual es desenvolupa el projecte i l'objectiu que es busca aconseguir, cal estudiar una mica el mercat actual, per tal de veure què ens ofereix, quines eines es poden fer servir i, en cas d'haver-ho de menester, quines parts s'hauràn de desenvolupar en la seva totalitat.\\
\newline D'aquesta tasca, en neixeràn les diferents alternatives a partir de les quals, s'haurà de fer una tria que acabarà per determinar el rumb a seguir del projecte.\\
\newline Finalment, les possibles opcions (amb els seus itineraris marcats) s poden reduïr a tres:
\begin{itemize}
	\item Lleida.net
	\item Logalty
	\item Desenvolupament pròpi basat en Blockchain i OTP.
\end{itemize}
S'ha trobat una quarta opció, Sage Bionetworks, però per temes que veurem més endavant, queda automàticament descartada.\\
\newline A la secció \textit{Valoració d'alternatives} (\ref{sec:validacio}) d'aquest mateix document, es tractarà el tema amb més profunditat.

\subsection{Eines i frameworks}
El desenvolupament de la plataforma es fa amb dos coneguts frameworks, Symfony i AngularJS.\\
Donada la desconeixença de les tecnologíes emprades per al desenvolupament general de la plataforma, cal un petit procés d'habituallament a l'entorn de desenvolupament.

\subsection{Disseny UI/UX del mòdul}
Paral·lelament a tot l'estudi

\subsection{Desenvolupament}

\subsection{Validació i testeig}

\subsection{Integració}

\subsection{Documentació}