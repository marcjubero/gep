\section{Context}

Aquest Treball de Final de Grau (d'ara en endevant \textbf{TFG}) en modalitat B, es desenvolupa a l'empresa Made of Genes \footnote{http://www.madeofgenes.com} com a pràctiques curriculars de l'estudiant.
\newline Per entendre el perquè d'aquest TFG, cal tenir en compte tres coses. Primerament, l'empresa en la qual s'ha desenvolupat el projecte. 
\newline Segon, s'ha de tenir clar el terme de \textit{consentiment informat} i, finalment, ser conscient de les metodologíes més emprades actualment per la signatura electrònica de documents.

\subsection{L'empresa, Made of Genes}
\textit{Made of Genes} és una empresa que ofereix un servei de genòmica personalitzada que posa a l'abast dels usuaris la seqüenciació del seu genoma i guardar-ne la informació de forma segura i de per vida.\\
Per altre banda, l'empresa ofereix una plataforma online que actua com a \textit{marketplace} on es poden comprar aplicacions de terceres parts basades en el genoma.
Aquestes aplicacions estàn disponibles per que aquelles persones que hagin contractat el servei de seqüenciació puguin treure partit de les dades enmagatzemades.\\
\newline La compra d'aquestes aplicacions/serveis, però, implica que les dades genòmiques dels usuaris són cedides a  tercers, i que aquests, amb les dades respondràn als serveis contractats pels usuaris. 
\newline Per assegurar que aquest porcés sigui lícit, el pacient ha de ser conscient de què és el que està contractant i què implica la contractació del esmentat servei. Per això es fa ús del consentiment informat.

\subsubsection{Rols a la plataforma}
\begin{itemize}
	\item \textbf{Pacient}: L'usuari final, aquella persona que compra el servei de seqüenciació juntament amb una o varies aplicacions sobre les dades del genoma.
	\item \textbf{Professional sanitari}: El professional que facilitarà la informació, tant la relativa al consentiment informat com la dels resultats del servei adquirit, a l'usuari final. Farà d'intermediari entre l'analista i l'usuari.
	\item \textbf{Analista}: Aquell professional sanitari que farà ús de les dades cedides per l'usuari, en realitzarà els anàlisis i presentarà al professional mèdic un informe dels resultats.
\end{itemize}

\subsection{Consentiment informat}
En l'àmbit mèdic, rep el nom de \textit{\textbf{consentiment informat}} el procediment a través del qual es garantitza que un pacient expressa de forma voluntària la intenció de participar en una investigació o tractament, havent prèviament comprès la informació que se li ha facilitat sobre l'estudi o tractament a realitzar, així com els beneficis, posibles riscos i alternatives i els seus drets i deures.\\
\newline En ocasions, i en contextos poc rellevants com podria ser un exàmen físic, aquest consentiment es pot arribar a sobreentendre i no requerir la presencia d'un document. No obstant, en procediments invasius, que impliquin cert nivell de risc o bé amb alternatives, el consentiment informat s'ha de presentar per escrit i ha de ser signat pel pacient.\\
\newline Aquest document, serveix per autoritzar a les organitzacions, metges o professionals sanitaris en general, a dur a terme les operacions necessàries amb la seguretat de que el pacient, o la persona sobre la qual recaigui l'efecte del tractament o investigació, n'és conscient.

\subsection{Sobre signatures}
La llei 59/2003 article 1, paràgraf 1, defineix:% la signatura electrònica com el conjunt de dades en format electrònic, que poden ser emprats com a medi d'identificació del firmant.\\
\begin{displayquote}
\textit{La firma electrónica es el conjunto de datos en forma electrónica, consignados junto a otros o asociados con ellos, que pueden ser utilizados como medio de identificación del firmante.}
\end{displayquote}

Alhora, en defineix també 3 modalitats:
\begin{itemize}
	\item \textbf{Signatura electrònica}: Correspon literalment a la definició anterior.
	\item \textbf{Signatura electrònica avançada}: és aquella signatura que permet identificar al firmant alhora que permet identificar qualsevol canvi en les dades del signant. Aquesta signatura ha estat generada amb mètodes que el signant pot mantenir sota el seu control exclusiu.
	\item \textbf{Signatura electrònica reconeguda}: Correspon a la signatura avançada però en aquest cas, basada en un certificat reconegut i generada mitjançant un dispositiu segur de creació de signatures.
\end{itemize}

La anterior llei també estipula, en el quart paràgref del mateix article, que la signatura electrònica reconeguda té el mateix valor que la signatua manuscrita.

Prenent la tercera tipología de signatura electrònica, la que estipula una signatura creada mitjançant dispositius segurs, i amb la creixent necessitat de garantir la validesa i legalitat de tràmits de diferents tipus a Internet, neix el concepte de tercer de confiança.\\
\newline El tercer de confiança, intentant buscar un paral·lelisme cotidià, es podria entendre com un notari que certifica que en un document, o en el seu defecte un tràmit, es va expedir (o efectuar) en un moment i amb un contingut determinats, però en aquest cas, el notari és una entitat que es troba a l'altre costat del cable de xarxa.\\
\newline Aquestes entitats, fortament regulades per la llei, conten amb certificats totalment vàl·lids i reconeguts xpedits per una entitat certificadora superior, que juntament amb un segell de temps, poden garantir el no repudi del documents signats, així com oferir mètodes per tal de que els usauris en puguin validar la integritat.

\subsection{One Time Password}
Rep el nom de \textit{One Time Password}, o còdi únic, aquell còdi generat en un moment concret i que té una validesa relativament curta que oscil·la entre els escassos minuts i les hores. Un cop expira aquest lapse de temps, cal generar-ne un de nou.\\
\newline Per tal de generar aquests codis únics, el sistema es basa generalment, existeix un segon mètode de generació via algorismes matemàtics, en el càlcul de codi prenent com a base l'instant de temps en el qual s'ha fet a petició i en un "secret" per cada usuari; l'esmentat "secret" correspon a una cadena de caràcters aleatoris que serà diferent per a cada usuari, garantint d'aquesta manera, que per un instant de temps i un usuari concret, el codi generat serà únic.\\
\newline El sistema de OTP és un sistema bastant extès dins de les entitats certificadores que permeten signatura online.

\subsection{Blockchain}
\textit{Blockchain} és un concepte o tecnología aparegut en els darrers anys que neix de forma conjunta amb el concepte de les criptomonedes.
És una tecnología totalment distribuïda, i que recentment està trobant aplicacions en diferents sectors; un dels quals és el de la verificació de documents mitjançant la publicació del hash d'aquests a la \textit{blockchain}.\\
\newline Al llarg del document s'explorarà amb més profunditat aquest concepte, així com els usos.




%Si busquem a un diccionari com podria ser el de la \textit{RAE\footnote{http://dle.rae.es/index.html}} la definició de signatura (firma en castellà) trobariem, entre altres, la següent:
%\begin{displayquote}
%\textit{"Nombre y apellidos escritos por una persona de su propia mano en un documento, con o sin rúbrica, para darle autenticidad o mostrar la aprobación de su contenido."}
%\end{displayquote}
%De la definició anterior, cal destacar la part que diu "\textbf{escritos por una persona de su propia mano}"; i es que en ple segle XXI, el mètode que més poder té a nivell legal per a signar documents, és la firma manuscrita.\\
%Són diversos els mètodes que intenten adaptar-se als nous temps adaptant els diferents avenços tecnològics que sorgeixen, però cap ha aconseguit la acceptació  jurídica de la que gaudeix la firma manuscrita.\\
%\newline Darrerament ha aparegut en el nostre país el que s'anomena e-DNI o DNI electrònic. Aquest intenta aportar una base legal (aportada directament des el govern central) a tot el dilema de la signatura digital, però ja sigui per la manca de confiança de la gent de peu o per la desconeixença generalitzada de la matèria, és evident que no acaba de funcionar.\\
%\newline Més tocant a la branca que ens ocupa, la de la informàtica, ja fa uns anys que ha aparescut un terme anomenat \textit{blockchain} que sembla que gaudeix de cert grau d'aprovació i confiança i que amb el temps va sonant cada cop més i en camps més diversos.











