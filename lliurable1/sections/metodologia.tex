\section{Metodología i rigor}

\subsection{Organització de l'equip}

Per al desenvolupmanet del projecte s'adoptaràn les metodologíes emprades a l'empresa, que en aquest cas, són el que s'anomenen metodologíes àgils; conretament l'anomenada \textit{Scrum\footnote{http://scrummethodology.com}}.\\
Scrum es basa en la realització d'iteracions durant el procés de desenvolupament que reven el nom d'\textit{sprints}. Les esmentades iteracions es composen d'un seguit de tasques que s'han de completar al llarg de la durada dels \textit{sprints}; que acostuma a oscil·lar entre una setmana i un mes.\\
Els objectius a assolir durant l'\textit{sprint} es fixen en unes reunions que es duen a terme l'inici anomenades \textit{Sprint Planning Meeting}.\\
Per altre banda, durant l'exercici de l'\textit{sprint} es realitzen reunions periòdiques anomenades \textit{Daily Scrum Meetings}, on es tracta de respondre a les següents preguntes:

\begin{itemize}
	\item Què vaig fer ahir?
	\item Què faré avui?
	\item Quins impediments he trobat fins ara?
\end{itemize}
Les anteriors preguntes intenten donar una visió el més àmplia possible de l'estat del projecte a tots els memebres de l'equip, així com permetre la resolució col·laborativa dels diferents problemes que vagin apareixent al llarg del desenvolupament.\\
\textit{Scrum} permet reaccionar de forma àgil a les diferents alteracions que poden sorgir al llarg del desenvolupament i que els desenvolupadors realitzin els canvis pertinents.\\
Al tractar-se d'un equip de desenvolupament reduït on la comunicació entre membres és constant, el seguiment de la filosofía \textit{scrum} resulta a vegadas un tant òbvia. Tot i aixó, es respecten les \textit{daylies} a l'inici de cada jornada i els \textit{sprint plannings} a l'inici de cada iteració.

\subsection{Codi i control de versions}

El desenvolupament del codi del projecte es divideix en dos parts clarament diferenciades:
\begin{itemize}
	\item \textbf{Backend}: desenvolupat amb Symfony, un dels frameworks PHP més extesos dins de la comunitat PHP per la seva versatilitat i potència.
	\item \textbf{Frontend}: desenvolupat amb AngularJS, un framework suportat per Google i  que recentment ha alliberat la release final de la versió 2. Actualment, es postula com un dels principals frameworks 
\end{itemize}

Pel control de versions es fa ús de \textit{Git}, un sistema de control de versions desenvolupat primerament per Linus Torvalds (creador del kernel de Linux) i que gràcies a plataformes com \textit{GitHub} o \textit{Bitbucket} s'ha convertit en un dels sistemes de control de versions més emprat en el món del desenvolupament de software.\\
\newline Per altre banda, per tal d'organitzar el flux de treball dins del repositori, s'ha decidit seguir un esquema com és \textit{Gitflow\footnote{http://nvie.com/posts/a-successful-git-branching-model/}}.\\
A mode de resum, \textit{gitflow} proposa una organització dins del repositori molt clara i estructurada. \\
\newline La estructura bàsica del repositori contarà amb dos branques principals:
\begin{itemize}
	\item \textbf{Master}: en termes més autòctons, la branca de producció. En aquesta branca del repositori sols hi ha codi plenament funcional i testejat. El codi que aqui es troba, està preprat per a ser publicat en qüalsevol moment.
	\item \textbf{Develop}: aquesta és la branca que es destinarà al desenvolupament del còdi pròpiament dit. El codi que aqui es trobi, també haurà d'estar testejat i ésser funcional, però el grau de rigurositat d'questa branca és menos que master.
\end{itemize}
Un cop definit el punt de partida, es crearàn branques a partir de la branca desenvolupament per a les diferents funcionalitats, aquestes rebràn el nom de \textit{feature} seguit del codi de la tasca a desenvolupar. D'aquesta manera les branques queden etiquetades i vinculades amb la tasca.
\newline Un cop acabada una tasca, es farà \textit{merge} de la branca \textit{feature-X} cap a la branca de desenvolupament i així successivament.
\newline Un cop acabat l'\textit{sprint}, es farà \textit{merge} de la branca desenvolupament cap a la branca \textit{master}. Aquest procediment l'anomenarem \textit{release}.
\newline Per a possibles correccions d'última hora, \textit{gitflow} proposa una quarta branca anomenada \textit{bugfix}. Aquesta surt directament de la branca \textit{master}, i està pensada per a correccions de codi ràpides.\\
Per tal de mantenir el repositor el més net possible, cada cop que s'acabi una \textit{feature} o un \textit{bugfix} s'ha de tancar la branca creada.

\subsection{Software de suport}
Com a suport per a la gestió d'\textit{Scrum} i del control de versions, així com de documentació interna, es disposa de llicència de la suite d'\textit{Atlassian\footnote{https://www.atlassian.com}}, que ofereix diferents aplicatius:
\begin{itemize}
	\item \textbf{Jira}: per a la gestió d tasques i planificació de les iteracions.
	\item \textbf{Confluence}: per a la documentació interna del projecte.
	\item \textbf{Bitbucket}: com a repositori de control de versions.
\end{itemize}
Per altre banda, també es fa servir \textit{Jenkins\footnote{https://jenkins.io/}} per a la integració contínua.

\subsection{Validació}
El mètode de validació que se segueix va lligat amb la metodología \textit{Scrum}.\\
\textit{Scrum}, defineix una série d'iteracions a partir de les quals s'organitza la feina a relaitzar al llarga de tot el projecte. En acabar cada una d'aquestes iteracions, per a que el codi sigui acceptat,  aquest ha de ser totalment funcional i presentar-se amb una bateria de tests que garantitzin el seu bon funcionament.

















