\section{Implementació}
\subsection{Entorn de desenvolupament}
Per dur a terme el desenvoupament s'ha fet ús, principalment, dels IDE\cite{ide} de l'empresa JetBrains\cite{jetbrains}:
\begin{itemize}
	\item Webstorm, per al desenvolupament de la part client (frontend)
	\item PHPStorm, IDE per a PHP per al desenvolupament de la part servidora
\end{itemize}
Un IDE, a diferència d'un editor de text normal que es podrien entendre com a eines de propòsit general, és una eina preparada per a treballar amb una tecnología concreta, que integra tot un seguit de complements o eines addicionals que busquen agilitzar el desenvolupament amb una determinada tecnología.\\
En el cas concret dels IDE emprats, Webstorm està pensat per treballar amb javascript, mentre que PHPStorm, tal i com es pot entreveure pel nom, està pensat per treballar amb PHP.\\
Tot les particularitats de cada un dels IDE, molts cops permeten extendre les seves funionalitats cap a altres llenguatges i/o frameworks amb l'us de complements.
\subsection{Gestió de projecte}
Per tal de gestionar les tasques i les diferents iteracions marcades per l'ús dela metodología àgil \textit{Scrum}\cite{scrum} es fa ús de la suite d'aplicacions d'Atlassian\cite{atlassian} que integra tot un seguit d'eines que permeten la getió de projectes:
\begin{itemize}
	\item \textbf{Jira} per a la gestió de les tasques i \textit{sprints}.
	\item \textit{Confluence} per a la documentació general del projecte.
	\item \textit{Bitbucket} com a repositori de control de versions.
\end{itemize}
Per al versionat de codi, es fa servir Git\cite{git} com a motor de control de versions i se segueix la metodología GitFlow\cite{gitflow} per tal de mantenir el repositori organitzat d'una forma coherent.