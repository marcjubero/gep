\section{Especificació de requisits}
%- Descripció i motivació del projecte i del entorn en que se desenvoluparà el projecte, e.g. si es en una empresa, una descripció de la empresa, si es un grup d’investigació de la universitat, el tipus de grup i activitats que realitza, etc. En que aspectes el projecte es relaciona amb la especialitat a la que pertany.

%- Estat actual, si es cau del projecte, e.g si es una millora de una infraestructura existent, hi ha que descriure el estat de la infraestructura, si es part de un projecte mes gran y existent, una descripció curta del encaix en el projecte global, etc. 
El consentiment informat és un procediment que, dins de l'àmbit mèdic, és pràcticament obligatori i serveix per a que els professionals mèdics informin als pacients d'un tractament i/o estudi sobre diferents aspectes del relacionats amb el procediment; aquesta informació és d'extrema rellevància i cal que el pacient en sigui conscient. Els temes que es tracten en el procés son tan diversos com riscos existents, alternatives, possibles millores resultants, etc.\\
\newline El procés de consentiment informat, culmina amb l'emissió d'un document per part del profesional mèdic i la posterior signatura d'aquest per part del pacient.\\
Amb aquesta signatura es posa en manifest que el pacient ha rebut la informació necessària per a tenir coneixement del tractament/estudi, que aquest mateix ha entès la informació que se li ha facilitat i que, de forma totalment voluntària i amb ple ús de les seves facultats, hi està d'acord.\\
\newline Tal i com es pot suposar de les línes anteriors, el procés del consentiment informat és quelcom totalment "manual" que requereix de la interacció dos actors.\\
Per una banda un emissor que emet un missatge; en aquest cas, el professional mèdic i, per l'altre, un receptor, que en aquest cas és pacient que se sotmet a un tractament o participa en un estudi.\\
\newline A dia d'avui, en un entorn mèdic convencional, aquest és el procediment que se segueix. No obstant, amb l'aparició de les noves tecnologíes i, principalment amb l'aparició d'internet era qüestió de temps que aparegués la necessitat de dotar al procediment de consentiment informat de capacitat telemàtica sense perdre validesa legal pel simple fet de que el procediment no compta amb una signatura firma manuscrita.\\
\newline És en aquest punt que apareix \textit{Made of Genes}\cite{mog}, una empresa de genòmica personalitzada que ofereix, per una banda desar les dades genómiques dels seus clients al núvol d'una forma segura, garantint que les dades són sempre de l'usuari i, per altre banda, una plataforma web a mode de \textit{marketplace} on els usuaris poden comprar serveis basats en les seves dades genòmiques.\\
\newline Aquestes dades desades al núvol tenen la categoría de dades clíniques i, per tant, la llei obliga l'ús de del consentiment informat; en aquest cas concret, per a que els ususaris que compren els serveis disponibles a la plataforma permetin que, aquells tercers que els ofereixen, puguin accedir a les dades genòmiques desades.\\
\newline Queda patent doncs, la necessitat de dotar al consentiment informat d'una nova dimensió, per tal de que els usuaris de la plataforma \textit{Made of Genes}\cite{mog} puguin fer-ne ús sense necessitat de visitar al professional mèdic responsable de cada servei.\\
\newline És fruit d'aquesta necessitat que apreix el context en el qual es realitza aquest TFG, la necesitat de desenvolupar un mòdul dins de la plataforma que permeti, per una banda que els usuaris siguin capaços de  firmar els documents i permetre l'accés dels professionals mèdics a les seves dades genòmiques i, per l'altre, dotar de validesa legal i jurídica al procediment telemàtic de firma del consentiment informat.



