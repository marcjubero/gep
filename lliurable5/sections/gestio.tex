\section{Gestió de riscos}
Al començament del projecte no es poden determinar els riscos existents, per això es fa ús de \textit{Scrum}\cite{scrum} que ens permet, mitjançant les iteracions, no només organitzar les tasques per a facilitar el desencolupament, sino veure si hi ha cap desviació dins dels temps establerts en el projecte.\\
\newline Un cop detectada una desviació, donat que es tracta d'un projecte on el temps de durada està totalment fitat, es decideix juntament amb la resta de l'equip de desenvolupament de la plataforma, com actuar per tal de pal·liar la desviació detectada.\\
\newline Les possibles vies d'actuació són les següents:
\begin{itemize}
	\item Ampliació de l'equip de desenvolupament
	\item Aplaçar la data final de lliurament
	\item Reformular els requisits del projecte
\end{itemize}
De totes les opcions anteriors, la que es durà a terme en cas d'aver-hi desviacions dins dels plaços, serà la de reformular els requisits inicials del projecte, per tal d'intentar trobar aquelles caractaerístiques que no siguin essenials per al bon funcionament del projecte, assegurant d'aquesta forma que un cop assolida la data de finalització del projecte, s'haurà obtingut un mínim producte viable.
