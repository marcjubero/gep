\section{Control de gestió}
Atès a les metodologíes àgils que es fan servir dins de l'empresa per al desenvolupament de projectes, en aquest cas \textit{Scrum}, es poden detectar desviacions dins de les planificacions de forma molt ràpida i eficaç i actuar en conseqüència en un temps relativament baix.\\
\newline Per pal·liar les possibles desviacions que poguessin sorgir al llarg del desenvolupament, es contemplen tres possibles accions:
\begin{itemize}
	\item \textbf{Ampliació de l'equip de desenvolupament}\\
	\newline Aquesta opció, tal i com indica el seu nom, consisteix en la contractació de nous desenvolupadors, ja sigui amb un caràcter definitiu o temporal, per tal de reforçar l'equip i permetre assolir les diferents tasques que formen el projecte més ràpidament i amb una major solvència.\\

	\item \textbf{Aplaçar la data final de lliurament}\\
	\newline Al tractar-se d'un Treball de Final de Grau, la data d'entrega queda estipulada des d'un primer moment amb la intenció de fitar el projecte dins del quadrimestre corresponent; per altre banda, la finalització del conveni amb l'empresa també dictamina una data límit.\\
	
	\item \textbf{Reformular els requisits del projecte}\\
	\newline Finalment, la última opció que es planteja per tal de pal·liar possibles desviacions durant el transcurs del projecte, consisteix en reformular el plantejament inicial del que hauría de ser el projecte acabat, amb la intenció d'eliminar aquelles \textit{features} que no siguin estrictament necessàries i vitals per al bon funcionament d'aquest i que no siguin claus per a la seva posterior integració amb la plataforma.\\
\end{itemize}

De totes les opcions presentades, la que sembla que més viabilitat té dins dels plantejament que es dóna al projecte és la tecera, que planteja un reformulació del projecte, tenint sempre com a objectiu, la obtenció d'un mínim producte viable que es pugui integrar de forma satisfactòria dins de la plataforma.