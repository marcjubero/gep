\section{Informe de sostenibilitat}

Tal i com s'indica al document proporcionat des del moodle de l'asignatura; durant el transcurs de GEP, es realitzarà un primer informe de sostenibilitat, corresponent a una série de respostes pensades per a una fase inicial del projecte, que s'ampliarà amb la memòria final.

\subsection{Estudi de l'impacte ambiental}
\subsubsection{Consum del disseny}
No he estat capaç de trobar una resposta clara a les preguntes plantejades en aquest apartat de l'estudi de sostenibilitat.\\
El que em porta a pensar que al tractar-se d'un projecte que queda englobat dins d'un projecte ja existent i en funcionament perd una mica el sentit voler-ne fitar l'impacte.
\subsubsection{Petjada ecològica}
Actualment, tal i com es descriu en lliurables anteriors, el procediment relacionat amb el consentiment informat entre un professional mèdic i un pacient es realitza, generalment, de forma verbal i culmina amb la signatura manuscrita d'un document imprès a través del qual el pacient assegura haver entès i acceptat tota la informació facilitada pel professional.
Aquest procés és en gran part presencial (ja sigui la transmissió de la informació o la signatura del document final).\\
\newline La proposta d'aquest TFG, consisteix en la substitució d'aquest procediment presencial per un canal que permeti al pacient rebre la informació pertinent de forma telemàtica i, conseqüentment, en poder procedir a la signatura i acceptació del contingut del document final presentat a través de la plataforma; la qual cosa suposa una millora substancial en un procediment de per si bastant farregós.

\subsection{Estudi de l'impacte econòmic}
Actualment, tal i com s'explica en seccions anteriors, el procés del consentiment informat requereix d'un temps bastant elevat, donada la forna convencional de procedir.\\
El desenvolupament del projecte pretén millorar aquest procediment permetent als usuaris canviar els contractes en paper per a documents PDF que podrà signar de forma electrònica, evitant així desplaçaments i impressions de documents innecesaris.

\subsection{Estudi de l'impacte Social}
\subsubsection{Impacte personal}
A nivell personal, el desenvolupament del projecte significa el descobriment de tecnologíes que fins el moment havia ignorat totalment (\textit{Blockchain}).\\ 
\newline Alhora, s'han ampliat temes relacionats amb alguns aspectes vistos al llarg de la carrera, com podría ser, per exemple, la validesa a nivell legal d'una signatura electrònica que permeten posar en context tot un seguit de coneixements adquirits però, que per diferens motius o bé per que no s'ha donat la ocasió, no usats.

\subsubsection{Impacte social}
A nivell social, considero que el desenvolupament del projecte pot aportar una millora de servei, ja no només el que s'ofereix a través de la plataforma \textit{Made of Genes}, sino que pot ajudar a millorar altres serveis i/o plataformes que facin ús del consentiment informat, ja que és fàcilment extrapolable a altres contextos, en general dins de l'àmbit mèdic.\\
\newline A part de l'àmbit mèdic, el conjut de tecnologíes que s'estudia al llarg del projecte, poden tenir úsos força beneficiosos en altres àmbits.