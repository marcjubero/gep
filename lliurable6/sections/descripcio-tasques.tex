\section{Descripció de tasques}
La idea d'aquesta secció és la d'il·lustrar les tasques existents en un primer estadi del projecte.

\subsection{Aspectes legals de la signatua electrònica}
La idea d'aquesta tasca, és la de posar en context el projecte en tot al que fa referència a consentiments informats, validesa de signatures electròniques, què s'ha de tenir en compte per tal de que no es pugui repudiar un document, etc etc.

\subsection{Estudi tecnològic}
Un cop entès el context en el qual es desenvolupa el projecte i l'objectiu que es busca aconseguir, cal estudiar una mica el mercat actual, per tal de veure què ens ofereix, quines eines es poden fer servir i, en cas d'haver-ho de menester, quines parts s'hauran de desenvolupar en la seva totalitat.\\
\newline D'aquesta tasca, en neixeran les diferents alternatives a partir de les quals, s'haurà de fer una tria que acabarà per determinar el rumb a seguir del projecte.\\
\newline Finalment,  de les possibles opcions (amb els seus itineraris marcats) se n'haurà d'escollir una que marcarà definitivament el rumb a seguir dins del projecte.

\subsection{Eines i frameworks}
El desenvolupament de la plataforma es fa amb dos coneguts frameworks, \textit{Symfony}\cite{symfony} i \textit{AngularJS}\cite{angular}.\\
Donada la desconeixença de les tecnologies emprades per al desenvolupament general de la plataforma, cal un petit procés d'habituallament a l'entorn de desenvolupament.\\
Aquest procés d'aprenentatge es realitzarà durant la contextualització a nivell legal i l'estudi tecnològic mencionat a l'apartat anterior.

\subsection{Desenvolupament}
Un cop triades les tecnologies sobre les quals es basarà el projecte, i després del temps de d'aprenentatge en les tecnologies emprades per al desenvolupament de la plataforma principal, es pot procedir a començar el desenvolupament del projecte.\\
\newline Aquesta part és la que, evidentment, més temps requerirà.\\
Cal dir que abans de començar a "picar codi", i un cop identificats els requisits, cal planificar les diferents tasques i prioritzar-les, per tal de seguir el que dicta la metodologia \textit{Scrum}\cite{scrum}.

\subsection{Validació i testeig}
Durant tot el procés de desenvolupament, s'anirà validant i testejant tot el que es vagi desenvolupanet per tal de que tot vagi com s'espera; el procés de testeig i validació en permetrà identificar possibles mancances del disseny original i corretgir possibles errors que vagin sorgint durant el procés de desenvolupament, aconseguint d'aquesta forma que no passi res per alt.\\
\newline Aquesta tasca es  repetirà tants cops sigui necessari durant tot el procés de desenvolupament, per tal que el mòdul desenvolupat quedi com s'espera.

\subsection{Integració}
Com es porta dient des de bon principi, aquest projecte es tracta d'un mòdul, inicialment desenvolupat de forma independent, però que al finalitzar el seu desenvolupament, s'espera que s'integri amb el nucli de la plataforma, per tal d'oferir les fucionalitats desitjades.\\
\newline Doncs bé, un cop acabat el desenvolupament i haver superat la fase de validació i testeig, es procedirà a integrar el mòdul dins de la plataforma, realitzant les modificacions pertinents.

\subsection{Documentació}
Tot i no ser la part més important, la documentació sobre el per què de les decisions preses serà útil a l'hora de, en un futur, mantenir el codi desenvolupat.



