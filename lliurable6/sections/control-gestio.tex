\section{Control de gestió}
Com ja s'ha mencionat repetides vegades al llarg del document, per al desenvolupament del projecte es fa ús de la metodologia àgil \textit{Scrum}.\\
A mode de recordatori, sabem que \textit{Scrum} és una metodologia àgil que organitza les tasques a realitzar dins d'un projecte en el que s'anomenen \textit{sprints} o iteracions, i que aquestes es van succeïnt una després de l'altra fins acabar el projecte.\\
\newline Aquests \textit{sprints}, es basen en una planificació inicial consensuada per tot l'equip de desenvolupament durant les reunions que es porten a terme al començament.\\
En altres paraules, aquestes reunions inicials serveixen per determinar la feina que s'espera que es dugui a terme en el temps que duri l'\textit{sprint}.\\
\newline Si l'acabar l'\textit{sprint} hi ha tasques que no s'han pogut dur a terme, es pot actuar en conseqüència. Cal dir que no es considera una desviació el fet que un \textit{sprint} no es pugui completar totalment, sempre i quan es tracti d'un cas aïllat.\\
En el moment en que aquest fet es repeteixi vàries vegades, si que es considerarà una desviació.\\
%Atès a les metodologíes àgils que es fan servir dins de l'empresa per al desenvolupament de projectes, en aquest cas \textit{Scrum}, es poden detectar desviacions dins de les planificacions de forma molt ràpida i eficaç i actuar en conseqüència en un temps relativament baix.\\
